\section{Per-Account Logistic Regression Model for Propensity Prediction}
\label{sec:logit_model}

\subsection{Motivation}
In highly illiquid fixed-income markets, client trading activity is sparse and irregular, 
resulting in time series of binary interactions (\(a_{u,i,t} \in \{0,1\}\)) where a value of~1 
denotes that client \(u\) has shown interest (e.g., traded or requested a quote) in CUSIP~\(i\) at time~\(t\).
The objective is to estimate, for each client--CUSIP pair~\((u,i)\),
the probability that the client will show interest in that CUSIP at the next time step,
\[
\Pr(a_{u,i,t+1} = 1 \mid \text{historical interactions up to } t).
\]
To balance interpretability and parsimony, we model this probability using a logistic regression
applied separately for each client, with explanatory variables summarizing recent trading behaviour.

\subsection{Data Structure}
For each client \(u\), let \(X_u \in \{0,1\}^{T \times M}\) denote the interaction matrix over \(T\) trading days 
and \(M\) CUSIPs, where \(x_{u,i,t}=1\) if client \(u\) interacted with CUSIP~\(i\) on day~\(t\).
For a given target CUSIP \(i\), we construct a sequence of feature vectors 
\(\mathbf{x}_{u,i,t}\) summarizing the state of the client's trading activity up to day~\(t\),
and a binary target variable \(y_{u,i,t} = a_{u,i,t+1}\).

\subsection{Feature Construction}
Each feature vector \(\mathbf{x}_{u,i,t}\) is composed of two categories of variables:

\paragraph{CUSIP-specific recency features.}
\begin{itemize}
    \item \textbf{Recent activity indicator:} \(r^{(1)}_{u,i,t} = a_{u,i,t}\).
    \item \textbf{Short-term activity count:} \(r^{(5)}_{u,i,t} = \sum_{k=0}^{4} a_{u,i,t-k}\).
    \item \textbf{Medium-term activity count:} \(r^{(20)}_{u,i,t} = \sum_{k=0}^{19} a_{u,i,t-k}\).
    \item \textbf{Days since last trade:} \(d_{u,i,t} = \min\{k \ge 1 : a_{u,i,t-k}=1\}\), truncated at a fixed horizon.
\end{itemize}

\paragraph{Portfolio-level context features.}
\begin{itemize}
    \item \textbf{Total trades across all CUSIPs in the past $w$ days:}
          \(s^{(w)}_{u,t} = \sum_{i'=1}^{M} \sum_{k=0}^{w-1} a_{u,i',t-k}\).
    \item \textbf{Breadth of trading:} number of distinct CUSIPs traded in the last $w$ days.
    \item \textbf{Sector exposure:} proportion of recent trades in the same sector or rating bucket as CUSIP~\(i\).
    \item \textbf{Time-since-last-any-trade:} days since any activity by the client.
\end{itemize}

Calendar controls (e.g., day-of-week or month indicators) can also be added to capture seasonality.

\subsection{Model Specification}
For each client \(u\) and target CUSIP \(i\), 
we fit a logistic regression model of the form
\[
\Pr(a_{u,i,t+1}=1 \mid \mathbf{x}_{u,i,t})
= \sigma(\beta_{0} + \mathbf{x}_{u,i,t}^\top \boldsymbol{\beta}),
\]
where \(\sigma(z) = (1 + e^{-z})^{-1}\) is the logistic link function, 
\(\beta_{0}\) is an intercept term, and \(\boldsymbol{\beta}\) are regression coefficients.

The model is estimated via maximum likelihood with an $L_1$ or $L_2$ regularization penalty:
\[
\hat{\boldsymbol{\beta}} = 
\arg\min_{\boldsymbol{\beta}}
\left\{
 - \sum_{t} \big[ y_{u,i,t}\log p_{u,i,t} + (1 - y_{u,i,t})\log(1 - p_{u,i,t}) \big]
 + \lambda \|\boldsymbol{\beta}\|_{q}
\right\},
\]
where \(q=1\) (lasso) or \(q=2\) (ridge), and $\lambda$ controls regularization strength.
Class imbalance is handled through inverse-frequency weighting of the positive class.

\subsection{Interpretation and Output}
The fitted coefficients provide an interpretable mapping between recent activity patterns
and future trading propensity. 
Positive coefficients correspond to behavioural drivers that increase the probability of renewed interest,
while negative coefficients indicate lack of engagement or saturation.
The predicted probabilities \(\hat{p}_{u,i,t+1}\) can be used as ranking scores
to generate daily recommendation lists for each client.

\subsection{Implementation Notes}
\begin{itemize}
    \item The model is implemented using \texttt{scikit-learn}'s 
          \texttt{LogisticRegression} class with \texttt{class\_weight='balanced'}.
    \item Training is performed per account; hyperparameters $(\lambda, q)$ 
          can be tuned via time-series cross-validation.
    \item Feature computation and model training can be vectorized across CUSIPs 
          for scalability.
    \item Output probabilities are stored as $\hat{p}_{u,i,t+1}$ for downstream ranking or evaluation.
\end{itemize}
